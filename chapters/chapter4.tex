%----------------------------------------------------------------------
\section[Space Systems I]{Space Systems I}
%----------------------------------------------------------------------
 
%----------------------------------------------------------------------
\paragraph{Breakdown of a space system}
%----------------------------------------------------------------------

A space system consists of several segments working together, where at least
one of them is spaceborne. The common segments of a space system are:
%
\begin{enumerate}
\item The space segment: this is the spacecraft (or the various spacecraft) 
that are in orbit
\item The ground segment: the control centers and ground stations used to 
track, monitor, and command the space segment
\item The launcher segment: the launch vehicles used to set the space segment 
in orbit.
\end{enumerate}
%
Within the space segment, each spacecraft can be decomposed into the 
\textbf{payload} (i.e., the instruments, transponders, etc that the client 
wants to have in orbit) and the \textbf{spacecraft platform or bus} (
consisting of all the subsystems needed to support the operation of the 
payload and keep the satellite alive).

There are several spacecraft platform subsystems:
%
\begin{enumerate}
\item Power subsystem: in charge of producing, accumulating, distributing 
electric power to the different loads.
\item Telecommunications: uplink (telecommand) and downlink (telemetry and 
mission data) require of a communication subsystem to work.
\item Attitude determination and control subsystem: this subsystem determines 
the current orientation (attitude) of the spacecraft using sensors, and then 
changes it to the desired orientation using actuators like momentum wheels.
\item Propulsion subsystem: needed if the satellite must perform propulsive
maneuvers. Every now and then, small corrective maneuvers are 
required to maintain the same operational orbit, or to desaturate the ADCS.
\item Data handling: the on board computer must process the incoming commands 
from the ground segment, monitor and control all the other subsystems, produce 
the telemetry that will be sent to the ground, and store and sometimes process 
the mission data until it can be delivered to a ground station.
\item Structural subsystems: the structure is in charge of maintaining the 
geometrical distribution of the spacecraft, and enduring the stresses that 
occur during launch.
\item Thermal control subsystem: this subsystem must deal with the changing 
environmental conditions (sunlight/eclipse, different spacecraft orientations 
with respect to the Sun), and maintain the temperature of each component 
within the allowable ranges at all times.
\item Environmental control and life support: In case of manned missions, it 
is necessary to provide air, water, and food to the crew, keep temperature and 
humidity at reasonable levels, and to dispose or recycle the generated waste.
\end{enumerate}
 
%----------------------------------------------------------------------
\paragraph{Telecommunications: link budget equation}
%----------------------------------------------------------------------

When trying to reach the ground from a spacecraft, the signal in our 
communication must be sufficiently focused and have enough power for the 
ground antenna to be able to receive it and tell the signal apart from the 
background electromagnetic noise.

If we have determined that the minimum energy per bit of information for a successful communication link is $E_b$, we can write down the link budget equation as follows:
%
\begin{equation}
E_b \leq \frac{P L_t L_r L_a L_s G_t G_r}{R}
\end{equation}
%
where:
%
\begin{itemize}
\item $P$ is the power, i.e. energy per unit time, spent by the spacecraft in 
sending the signal
\item $L_t$ are any efficiency losses in the transmitter. E.g., if only $80\%$ 
of the input power eventually becomes the signal power, then $L_t=0.8$.
\item $L_r$ contemplates any similar losses at the receiving end of the link. 
\item $L_a$ are any losses that may exist due to the propagation of the signal 
through the atmosphere. The atmosphere may damp or absorb part of the signal 
power, meaning that only a fraction of the power makes it to the ground. The 
absorption depends on the wavelength of the signal, $\lambda$.
\item $L_s$ are the free space losses, which scale with the transmission 
distance $d$. Since the emitted power at the satellite expands outwards 
roughly spherically, the power per unit area decreases as $1/d^2$. $L_s$ is 
formally defined as $L_s=\lambda^2 /(4\pi d)^2$.
\item $G_t$ is the gain of the transmitting antenna, i.e., how well the 
antenna concentrates the radiated power in the desired direction, rather than 
emitting in all directions. It is defined as $G_t=4\pi A_t/\lambda^2$, 
where $A_t$ is the effective area of the antenna.
\item $G_r$ is the gain of the receiving antenna, i.e., how well the 
antenna selectively listens only to the desired direction, rather than all 
directions. It is defined as $G_r=4\pi A_r/\lambda^2$, where $A_r$ is the 
effective area of the antenna.
\item $R$ is the data rate of the transmission, i.e., how many bits per unit 
time are being sent.
\end{itemize}
%
The product $PL_t G_t$ is known as the \emph{effective isotropic radiated 
power} (EIRP).

%----------------------------------------------------------------------
\paragraph{Rotational dynamics: attitude determination and control}
%----------------------------------------------------------------------

The rotational dynamics of the spacecraft in orbit are governed by the angular 
momentum equation---the rotational analogous of the (linear) Newton's second 
law.
%
\begin{equation}
\frac{\dd \bm H_G}{\dd t} = \bm M_G
\end{equation}
%
where $\bm H_G$ is the total angular momentum of the spacecraft about its 
center of mass $G$, and $\bm M_G = \sum(\bm r - \bm r_G)\times \bm F$ the 
moment of all forces about the center of mass.

While this equation may seem complicated, the important aspect to be
understood in this course is that in the absence of external moments of forces
(i.e., if $\bm M_G = 0$), the total angular momentum of the spacecraft is
conserved. This means that we cannot change its state of rotation, unless the
spacecraft is composed of two or more bodies and we transmit angular momentum
from one body to another, or unless we use propulsion to change the total
angular momentum.

For instance, imagine a cubic spacecraft floating in empty space, with a wheel
in its interior. If the cube is not rotating initially, we can make it rotate
about the axis of the wheel if we spin up or down the wheel. Observe that
during the whole maneuver, the total angular momentum (sum of the angular
momentum of the cube and the wheel) remain constant.

This mechanism is used in the ADCS to control the orientation and rotation
state of the spacecraft in the presence of perturbations that create a moment
of forces. However, when the angular momentum stored in the wheels is too
large, the rotational speed of the wheels may reach its maximum, and we need
to use the propulsion system to \emph{desaturate} the wheels and start over.




